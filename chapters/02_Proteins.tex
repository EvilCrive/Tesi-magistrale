\chapter{An Overview on Proteins}
\label{chp:proteins}

\section{Basics of Molecular Biology}
\subsection{Cells}
Life is made of cells, the fundamental working units of every living system. They are composed of :
\begin{itemize}
	\item 70\% of water;
	\item 23\% of macromolecules, such as:
	\begin{itemize}
		\item proteins;
		\item polysaccharides.
	\end{itemize}
	
	\item 7\% of small molecules, such as:
	\begin{itemize}
		\item lipids;
		\item amino acids;
		\item nucleotides.
	\end{itemize}
\end{itemize}
Cells are the smallest structural unit of an organism capable of independent functioning. Each cell follows the same common cycle of birth, replication, protein synthesis and death. 

Cells have common features in their structure, in the next page we will show a figure, representing the main components of a cell.

\pagebreak

\begin{figure}[h!]
	\includegraphics[scale=.25]{res/proteins_overview/cell_structure.png}
	\centering
	\caption{Cell structure}
	\label{fig:cell-structure}
\end{figure}

Let's analyse the main components among the ones we can see from the figure \ref{fig:cell-structure}:
\begin{itemize}
	\item \textbf{Nucleus}: it's the main component of the cell which contains the DNA, the genetic information of the individual. Around the nucleus there is the nucleus membrane, not annotated in the figure, that separates the nucleus from the cytoplasm;
	\item \textbf{Golgi apparatus}: or Golgi vescicles. It's purpose is to help processing and packaging of molecules, especially proteins that will be exported from the cell;
	\item  \textbf{Mitochondria}: they are one of the most important cellular organelles, since they generate most of the chemical energy, the ATP, to power the cell's biochemical reactions. They generate the ATP through aerobic respiration;
	\item \textbf{Cytoplasm}: a gelatinous liquid that fills the inside of the cell, it's made of water, salts and various enzymes. It has many purposes: giving the cell a specific shape, protecting the cell organelles and assisting the cell's metabolic processes;
	\item \textbf{Cell membrane}: also called plasma membrane, is a sempipermeable membrane that separates the interior of the cell from the outside environment. This membrane regulates the trasport of molecules entering and exiting the cell, thanks to its semipermeability.
\end{itemize}
There are other elements we didn't describe in a cell, such as the endoplasmatic reticulum and some cell organelles such as peroxisome, lysosome, rybosome, secretory vescicles and others.
\subsection{DNA}
The nucleus of the cell contains the DNA, which is the genetic information of the individual. It might slightly change between cells, especially among different type of cells, such as muscle cells and brain cells, since their functionalities are different.
\subsubsection{Chromosomes}
The genome is an organism's complete set of DNA, the human genome contains about 3 billion DNA base pairs and 24 distinct chromosomes. 

\begin{figure}[h]
	\includegraphics[scale=0.8]{res/proteins_overview/chromosomes.png}
	\caption{Chromosomes set in human genome}
	\label{fig:chromosomes}
\end{figure}

In the figure \ref{fig:chromosomes} we can see the set of 24 chromosomes in the human genome. A cell normally contains 23 chromosomes, the 22 common ones, the autosomes, plus one of the sex chromosomes, "XX" in females and "XY" in males.
\subsubsection{Genes}
Each chromosome contains many genes, the basic functional units of heredity. Genes are specific sequences of DNA that encode instructions on how to make proteins.


\begin{wrapfigure}{l}{0.5\textwidth} %this figure will be at the right
	\centering
	\includegraphics[width=0.5\textwidth]{res/proteins_overview/genes.png}
\end{wrapfigure}

From the figure on the left we can see how genes, along with the DNA, fold into chromosomes, which will contain a really high density of DNA. And finally the chromosomes will be stored inside the nucleus of the cell.

Genes determine the principle hereditary stuff in the human being, from the height, to the growth, muscular mass and appetite. Some of these elements are defined by a group of genes, others from a singular gene. Mutations of some genes can modify the characteristics of a human, from the eye color to diseases. These mutations can also provide beneficial effects, such as immunity or protection for some diseases, for example there is a gene mutation that increase the protection against malaria.

Genes are vastly studied, especially their function in the DNA: some genes are highly related with diseases, such as:
\begin{itemize}
	\item Huntington disease;
	\item Down syndrome;
	\item Cystic fibrosis;
	\item Albinism;
	\item various types of cancer.
\end{itemize}

\subsubsection{DNA structure}

\begin{figure}[h!]
	\includegraphics[scale=.6]{res/proteins_overview/dna_basepairs.png}
	\centering
	\caption{DNA structure}
	\label{fig:dna-structure}
\end{figure}

The figure \ref{fig:dna-structure} represents the DNA structure, a double helix macromolecule composed of nucleotides. A nucleotide is a molecule composed of:
\begin{itemize}
	\item \textbf{Sugar phospate backbone}: a molecule of sugar combined with a group phospate;
	\item \textbf{Nitrogenous base}: there are four possible bases in DNA:
	\begin{itemize}
		\item Adenine;
		\item Thymine;
		\item Guanine;
		\item Cytosine.
	\end{itemize}
\end{itemize}
The nucleotides pair up with the nucleotides of the opposite helix, via the nitrogenous base: the only possible pairs are Adenine-Thymine and Guanine-Cytosine.

\pagebreak

The DNA can replicate itself or can be used to produce proteins, which are particular molecules that participate in most of the essential processes of the human body:
\begin{itemize}
	\item build and repair body structures;
	\item digest nutrients;
	\item hormones: some hormones are proteins or protein-derived. Hormones are chemical messengers that flow through blood to coordinate different body's functions;
	\item execute various metabolic functions, or assist the execution.
\end{itemize}
We will now see more in detail this process from DNA to proteins, called protein synthesis.

\subsection{Central Dogma of Biology}
\begin{figure}[h!]
	\includegraphics[scale=.27]{res/proteins_overview/central_dogma.png}
	\centering
	\caption{Central Dogma of Biology}
	\label{fig:central-dogma}
\end{figure}

The figure \ref{fig:central-dogma} represents the Central Dogma of Biology, which explains the flow of genetic information within a biological system: from DNA to RNA to protein.

\pagebreak

\subsubsection{Transcription}
Transcription is the process of copying a segment of DNA into RNA. An enzime called RNA polymerase produce an RNA strand, by passing through a DNA strand. The resulting RNA is the copy of the coding strand of the "input" DNA , while the strand used as template to create this copy is called template strand.

\begin{figure}[h!]
	\includegraphics[scale=.6]{res/proteins_overview/rna_polymerase.jpeg}
	\centering
	\caption{Central Dogma of Biology}
	\label{fig:transcription}
\end{figure}

In the above figure, we can see how RNA polymerase creates the RNA. We can note that in the figure the RNA is called messenger RNA, also known as \textbf{mRNA}.

There are other two types of RNA, tRNA and rRNA. Let's see the differences:
\begin{itemize}
	\item \textbf{mRNA}: messenger RNA, contains the genetic information from the DNA. mRNA specifies ;
	\item \textbf{tRNA}: transfer RNA, transfer the correct amino acid to the ribosome. It acts as a bridge between the mRNA and the ribosome;
	\item \textbf{rRNA}: ribosomal RNA, combined with ribosomal proteins creates the ribosome.
\end{itemize}
\subsubsection{Translation}
Translation is the process by which the genetic information from the DNA is converted into a functional protein. Its also known as protein synthesis.

The ribosome reads the next three nucleotides of the mRNA, this triplet is the \textbf{codon}, then the ribosome brings in the correct tRNA. The correct tRNA is the tRNA, whose triplet, the \textbf{anticodon}, matches the actual triplet of the mRNA.
Once we have a matched tRNA, the tRNA transfers the amino acid to the growing protein chain of the ribosome.

\begin{figure}[h!]
	\includegraphics[scale=.6]{res/proteins_overview/protein_synth.jpeg}
	\centering
	\caption{Protein Synthesization}
	\label{fig:protein-synth}
\end{figure}

In the figure \ref{fig:protein-synth} we can see how the ribosome, along with tRNAs identifies the correct amino acids and transfer it into the growing protein, through a chemical bond.

\pagebreak

In the table below, we can observe the Genetic Code: the set of triplets of nucleotides, codons, and the corresponding amino acid.

\begin{figure}[h!]
	\includegraphics[scale=1]{res/proteins_overview/genetic_code.png}
	\centering
	\label{fig:genetic-code}
\end{figure}

We can observe 4 particolar combinations of nucleotides that correspond to signal of start and stop the protein synthesization. The codon \textbf{AUG} identifies the START signal for the protein translation, while the codons \textbf{UAA, UAG, UGA} identify the STOP signal, this ends the translation process producing the final protein.

\pagebreak

\subsection{Proteins}
Proteins are large, complex molecules made up of amino acids, smaller subunits also referred to as residues. The residues are the building blocks of a macromolecule, so in this case the residues are the incorporated amino acids.
Proteins are linear chains of different combinations of 20 different amino acids. 

\subsubsection{Proteins' Functions}
\begin{itemize}
	\item Make up the cellular structure;
	\item Form body's major components, such as skin and hairs;
	\item Form hormones to communicate with other cells;
	\item Form enzymes to regulate gene activity.
\end{itemize}

Proteins work together with other proteins or nucleic acids, in highly specific lock-and-key ways. They fit together with high specificity.

The functions of proteins depends both on the amino acids order and types and on the three-dimensional structure the protein folds into.

\subsubsection{Protein Folding}
Proteins tend to fold into lowest energy three-dimensional conformation. They begin to fold already when the amino acid chain is being formed during translation.

The different amino acids have different chemical properties, by interacting with each other the protein starts to fold adopting its functional structure.

The structure of the protein is really important for the protein function:
\begin{itemize}
	\item Determines which amino acids are exposed outside the protein;
	\item Determines which substrates the protein can react with.
\end{itemize}

Substrates are molecules or compounds that participate in a chemical reaction, they are starting materials or reactants which are acted upon by enzymes or catalysts.

The tertiary structure or 3D structure is the global structure of the protein.

The secondary structure or 2D structure refers to local structural patterns the residues tend to form, through hydrogen bonds. THe main ones are :
\begin{itemize}
	\item alpha-helix: proteins bury most of its hydrophobic residues in the interior core, forming a spiral structure resembling an helical spring;
	\item beta-sheets: segments of the protein are stretched out and aligned in a sheet-like arrangement.
\end{itemize}
\subsubsection{Amino Acids}
Lastly, let's talk about the amino acids.
Amino acids are monomers of proteins, each amino acid has a specific chemical behaviour.
All amino acids in the human genetic code have a carboxyl (-COOH) and an amminic (-NHH) group bound to the alpha-carbon. They differ for the side chain. 
The various side chains differ for:
\begin{itemize}
	\item three-dimensional structure;
	\item electric charge;
	\item hydrophobicity.
\end{itemize}
% table of aminoacid name, polar/acidic/.., side chain , aliphatic/aromatic/...

\section{Computational Biology: Computer Science Applied to Biology}