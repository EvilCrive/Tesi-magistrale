\chapter{Introduction}
\label{chp:intro}

\section{BioComputing UP}
BioComputing UP is a research laboratory and is part of the Department of Biomedical Sciences of the University of Padua. The research there is focused on developing bioinformatic tools and high quality computational methods for important biological issues, main fields are: structural biology, functional biology, genome wide analysis, genetic diseases and cancer studies.
\section{Internship description}
The project of internship I carried on during the stage at BioComputing Up was mainly of expanding a pre-existent software tool, Alphafold-disorder, with newer libraries. In particular, my tasks were:
\begin{itemize}
    \item Studying Alphafold-disorder code and related papers, to understand the scientific principles beneath;
    \item Implementing an algorithm to detect protein secondary structure, based on atomic distances and angles, calculated with protein atomic coordinates and the library BioPython. This algorithm was described in a paper \cite{psea};
    \item Integrating this secondary structure detection algorithm into the software tool Alphafold-disorder;
    \item Developing a procedure to calculate the solvent-accessibility for each aminoacid in the protein, using the algorithm devised by Shrake and Rupley in 1973, by integrating a specific library;
    \item Integrating this procedure to compute solvent-accessibility into Alphafold-disorder;
    \item Integrating the library FoldComp and developing a quick procedure to allow input files of the type .fcz in Alphafold-disorder. FoldComp is a library that created .fcz files as a compression of normal protein data files, .pdb files. FoldComp library is described in a scientific paper \cite{foldcomp};
    \item Evaluating results through plots and machine learning models.
\end{itemize}

\section{Chapters' description}
This thesis is divided into chapters based on the 6 tasks I've done in my internship.

This first chapter is just a brief introduction.

The {\hyperref[chp:proteins]{second chapter}} is an introduction to proteins, with a focus on proteins with intrinsically disordered regions and the bionformatics library to interact with proteins, such as BioPython.

The {\hyperref[chp:alphafold-disorder]{third chapter}} is focused on the description of Alphafold-disorder and the relevant libraries involved, such as DSSP.

The [{\hyperref[cap:descrizione-architettura]{fourth chapter}}] describes the process of development of the main software application, the tool that detects the secondary structures of the various aminoacids based on protein atomic coordinates.

The [{\hyperref[cap:descrizione-architettura]{fifth chapter}}] describes the development of the procedure to compute solvent-accessibility and the integration into Alphafold-disorder.

The [{\hyperref[cap:descrizione-architettura]{sixth chapter}}] describes the integration of FoldComp library in Alphafold-disorder.

The [{\hyperref[cap:descrizione-architettura]{seventh chapter}}] describes the process of analysis of the results produced by the new software tool \underline{Alphafold-disorder (SASA)} through plots and machine learning models, to evaluate the quality of the results produced.

In the [{\hyperref[cap:descrizione-architettura]{last chapter}}], we will draw some conclusions based on the insights provided by the analysis with plots and ML models.

