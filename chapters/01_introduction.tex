\chapter{Introduction}
\label{chp:intro}

\section{BioComputing UP}
BioComputing UP is a research laboratory and is part of the Department of Biomedical Sciences of the University of Padua. The research there is focused on developing bioinformatic tools and high quality computational methods which are put to use to solve important biological issues. Its main fields are: structural biology, functional biology, genome wide analysis, genetic diseases and cancer studies.
\section{Internship description}
The internship project I carried on during the stage at BioComputing Up was mainly about expanding a pre-existent software tool, Alphafold-disorder, with newer libraries. In particular, my tasks were:
\begin{itemize}
    \item Studying Alphafold-disorder code and related papers, to understand the scientific principles beneath;
    \item Implementing an algorithm which detects protein secondary structure, based on atomic distances and angles, calculated with protein atomic coordinates and the library BioPython. This algorithm was described in a paper \cite{psea};
    \item Integrating this secondary structure detection algorithm into the software tool Alphafold-disorder;
    \item Developing a procedure to compute the solvent-accessibility for each aminoacid in the protein, using the algorithm devised by Shrake and Rupley in 1973, by integrating a specific library;
    \item Integrating this procedure to compute solvent-accessibility into Alphafold-disorder;
    \item Integrating the library FoldComp and developing a quick procedure to allow input files of the type .fcz in Alphafold-disorder. FoldComp is a library that creates .fcz files as a compression of normal protein data files, .pdb files. FoldComp library is described in a scientific paper \cite{foldcomp};
    \item Evaluating results through plots and machine learning models.
\end{itemize}

\section{Thesis Outline}
This thesis is divided into chapters based on the 6 tasks I've accomplished in my internship.

Chapter 1 : brief introduction and contextualization.

\underline{\hyperref[chp:proteins]{Chapter 2}}: overview on proteins, starting from the basics concepts of biology, with a focus on proteins with intrinsically disordered regions, to how computer science is applied in this field.

\underline{\hyperref[chp:alphafold-disorder]{Chapter 3}}: description of Alphafold-disorder and the relevant libraries involved, such as DSSP.

\underline{\hyperref[chp:development]{Chapter 4}}: development of the software tool AlphaFold-disorder (SASA). Focus on the three major procedures implemented.

%The \underline{\hyperref[cap:descrizione-architettura]{fourth chapter}} describes the process of development of the main software application, the tool that detects the secondary structures of the various aminoacids based on protein atomic coordinates.

%The [{\hyperref[cap:descrizione-architettura]{fifth chapter}}] describes the development of the procedure used to compute solvent-accessibility and its integration into Alphafold-disorder.

%The [{\hyperref[cap:descrizione-architettura]{sixth chapter}}] describes the integration of FoldComp library in Alphafold-disorder.

\underline{\hyperref[chp:analysis]{Chapter 5}}: analysis of the results produced by the new software tool Alphafold-disorder (SASA) through plots and machine learning models, to assessment of the results produced.

\underline{\hyperref[chp:conclusions]{Chapter 6}}: conclusions based on the insights provided by the plots made in the Chapter 5.

