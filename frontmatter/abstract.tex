%!TEX root = ../main.tex
\begin{abstract}
In this thesis we describe the work that I, Alberto Crivellari, along with the professors Damiano Piovesan and Alexander Monzon, carried out during my internship period. 

We developed a variation of the software tool AlphaFold-disorder: AlphaFold-disorder (SASA), by implementing PSEA and SASA algorithms. Then we compared the quality of results of the new software tool with the original one.

Regarding the development we made three major procedures: the PSEA prcoedure to predict secondary structures, based on three-dimensional coordinates of amino acids; the SASA procedure to compute RSA of amino acids, using the SASA library; the FoldComp procedure for managing .fcz files, which are compressed protein files.

To assess the quality of the results we first plotted the dataset to gain insights on the features' distribution and then we implemented ML models. Finally we compared ROC and Precision-Recall curves between AlphaFold-disorder, AlphaFold-disorder (SASA) and the best ML model to obtain some insight on the results: AlphaFold-disorder (SASA) prediction are on par to AlphaFold-disorder ones, while the ML model needs more training data to surpass their predictions.

\end{abstract}