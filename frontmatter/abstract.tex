%!TEX root = ../main.tex
\begin{abstract}
Structural Bioinformatics is a branch of science which involves the analysis 
of three-dimensional structures of molecules. One of the main molecules considered are proteins,
 at the beginning the focus was on proteins with a fixed three-dimensional structure.
But recently, the researchers are shifting the focus to IDPs, Intrinsically Disordered Proteins,
 which are proteins that have disordered regions: parts of the protein doesn't belong to any fixed conformation, but instead they have highly flexible conformations.
The work we did on this thesis is focused on recognizing IDPs, through the extraction of features which are indicators of disorder. 

A variation of the software tool AlphaFold-disorder, named AlphaFold-disorder (SASA), was developed by implementing PSEA and SASA algorithms. Subsequently, the quality of results produced by the new software tool was compared with the original one.

The development process involved three major procedures: the implementation of the PSEA procedure for predicting secondary structures based on three-dimensional coordinates of amino acids; the implementation of the SASA procedure for computing RSA (Relative Solvent Accessibility) of amino acids using the SASA library; and the implementation of the FoldComp procedure for managing .fcz files, which are compressed protein files.

To assess the quality of the results, the dataset was initially plotted to gain insights into the distribution of features. Machine learning models were then implemented. Finally, ROC and Precision-Recall curves between AlphaFold-disorder, AlphaFold-disorder (SASA), and the best machine learning model were compared. The comparison revealed that AlphaFold-disorder (SASA) predictions are on par with AlphaFold-disorder ones, while the machine learning model requires more training data to surpass their predictions.

\end{abstract}